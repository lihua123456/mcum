\documentclass[fontset=windows,a4paper,12pt]{ctexart}
\usepackage{geometry}
\usepackage{graphicx}
\usepackage{subfigure}
\usepackage{graphics}
\usepackage{overcite}
\usepackage{listings}
\usepackage{CJK}

\pagestyle{plain}
\usepackage{bm}
\usepackage{amsmath}
\usepackage{multirow}

\geometry{top=25mm,bottom=20mm,left=25mm,right=25mm}
\renewcommand*{\baselinestretch}{1.38}

\begin{document}
  \begin{center}
  	\zihao{3}{\heiti 小区开放对道路通行的影响}
  \end{center}
  \linespread{1.2}
  \begin{center}
  	\zihao{4}{\heiti 摘\ \ \ \ 要}
  \end{center}
  \zihao{4}{\songti 
  	
  	摘要摘要摘要摘要
  	摘要摘要摘要摘要
  	摘要摘要摘要摘要
  	摘要摘要摘要摘要
  	摘要摘要摘要摘要

  }
  \textbf{关键词:}\ 层次分析\ 元胞自动机
  
  \section{问题重述}
  \section{符号说明}
  \section{问题分析}
	\subsection{问题一}
		考虑小区开放对周边道路通行的三个影响因素:道路条件、交通条件。
		\subsubsection{道路条件}
			经过对小区周边道路网的分析得出,道路条件受以下两个因素影响:延误时间、行程时间。

			1. 延误时间(再加上道路信息:车道数目、道路交叉口数目的影响)
			\begin{equation}
				d_1=\frac{0.5T(1-\frac{t_g}{T})}{1-[min(1,x)\cdot{\frac{t_g}{T}}]}
			\end{equation}
		\subsubsection{行程时间}
			基于美国联邦公路局函数(即BPR函数) ,并对其改进,考虑到小区内道路上行人、自行车等非
			机动车较多的特点,增加行人对机动车的影响、自行车对机动车的影响。基于已有的研究成果
			,得到行人、自行车分别对车辆的影响系数,得到改进BPR函数。
	\subsection{问题二}
  \newpage
  \bibliography{mybib}
  \bibliographystyle{gbt7714}

\end{document}